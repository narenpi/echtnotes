% \PassOptionsToPackage{table}{xcolor}
\documentclass[notitlepage, 12pt]{article}
\date{}
\usepackage{titlesec}
\titleformat{\chapter}[display]
{\normalfont\bfseries}{}{0pt}{\Large}

\usepackage{titling}   
\usepackage{marginnote}
\usepackage[paper=a4paper,top=1in, left=1in, right=1in, bottom=1in,  heightrounded,
marginparwidth=1in, marginparsep=3mm]{geometry}
\usepackage[dvipsnames]{xcolor}

\usepackage{amsthm, amssymb, amsfonts, amsmath}
\usepackage{graphicx, tikz, hyperref, enumitem, mathrsfs, tikz-cd, adjustbox, xfrac, stackrel,MnSymbol, multicol }
\usetikzlibrary{calc,shapes}
\usepackage[fleqn,tbtags]{mathtools}

\usepackage{nameref}
\usepackage{wrapfig}
\usepackage{faktor}
\usepackage{bbold}
\usepackage{float}
\usepackage{todonotes}
\usepackage{parskip}
\usepackage{cleveref}
\usepackage{eucal}

\newcommand{\C}{\mathbb{C}}
\newcommand{\R}{\mathbb{R}}
\newcommand{\Q}{\mathbb{Q}}
\newcommand{\hh}{\mathcal{H}}
\newcommand{\Z}{\mathbb{Z}}
\newcommand{\N}{\mathbb{N}}
\newcommand{\G}{\mathcal{G}}
\newcommand{\kx}{k[X_n]}
\DeclareMathOperator{\depth}{depth}
\theoremstyle{definition}
\newtheorem{theorem}{Theorem}[section]
\newtheorem{corollary}[theorem]{Corollary}
\newtheorem{lemma}[theorem]{Lemma}
\newtheorem{definition}[theorem]{Definition}
\newtheorem{example}[theorem]{Example}
\newtheorem{remark}[theorem]{Remark}
\newtheorem*{claim}{Claim}
\newtheorem{proposition}[theorem]{Proposition}

\newcommand{\rnc}[1]
    {\MakeUppercase{\romannumeral #1}}
\DeclareMathOperator{\im}{im}
\DeclareMathOperator{\war}{wr}
\DeclareMathOperator{\lt}{lt}
\DeclareMathOperator{\mon}{Mon}

% Set link colors for labeled items (blue), citations (red), URLs (orange)
\hypersetup{colorlinks=true, linkcolor=Red, citecolor=RedOrange, urlcolor=ForestGreen}

\makeatletter
\newtheoremstyle{para}
  {\topsep}   % ABOVESPACE
  {\topsep}   % BELOWSPACE
  {\upshape}  % BODYFONT
  {0pt}       % INDENT (empty value is the same as 0pt)
  {\bfseries} % HEADFONT
  {.}         % HEADPUNCT
  {5pt plus 1pt minus 1pt} % HEADSPACE
  {\thmnumber{#2}\@ifnotempty{#3}{ \thmnote{#3}}} % CUSTOM-HEAD-SPEC
\makeatother

\theoremstyle{para}
\newtheorem{para}[theorem]{}
\newtheorem*{para*}{para}
\makeatletter
\newcommand{\namelabel}[1]{%
  \phantomsection
  \renewcommand{\@currentlabel}{#1}% Update the label text/name
  \label{#1}% Set the label
}
\makeatother

\usepackage{abstract}
\renewcommand{\abstractname}{}    % clear the title
\renewcommand{\absnamepos}{empty}

\usepackage[T1]{fontenc}
% \usepackage{newtxtext,newtxmath}
\usepackage{libertine,libertinust1math}
\usepackage[most]{tcolorbox}
\makeatletter
\newcommand*{\toccontents}{\@starttoc{toc}}
\makeatother
%\usepackage[nottoc,notlot,notlof]{tocbibind}
\usepackage{etoolbox}


\newcommand{\leftrarrows}{\mathrel{\raise.75ex\hbox{\oalign{%
  $\scriptstyle\leftarrow$\cr
  \vrule width0pt height.5ex$\hfil\scriptstyle\relbar$\cr}}}}
\newcommand{\lrightarrows}{\mathrel{\raise.75ex\hbox{\oalign{%
  $\scriptstyle\relbar$\hfil\cr
  $\scriptstyle\vrule width0pt height.5ex\smash\rightarrow$\cr}}}}
\newcommand{\Rrelbar}{\mathrel{\raise.75ex\hbox{\oalign{%
  $\scriptstyle\relbar$\cr
  \vrule width0pt height.5ex$\scriptstyle\relbar$}}}}
\newcommand{\longleftrightarrows}{\leftrarrows\joinrel\Rrelbar\joinrel\lrightarrows}

\makeatletter
\def\leftrightarrowsfill@{\arrowfill@\leftrarrows\Rrelbar\lrightarrows}
\newcommand{\xleftrightarrows}[2][]{\ext@arrow 3399\leftrightarrowsfill@{#1}{#2}}
\makeatother

\begin{document}
\begin{titlepage}
    \title{eCHT course on Stable Homotopy Theory
}
\author{\normalsize Naren\\
\normalsize University of Kentucky\\
\normalsize Spring 2024}
	\maketitle
	\thispagestyle{empty}
    	\toccontents
\end{titlepage}
\pagebreak
\marginnote{L3, 23/1}
\section{Homotpy limits and colimits }
\subsection{Motivation}
Notation: $\mathcal{C}$ homotopical category. $\mathcal{M}$ model category. 

Usual limits colimts use $1$-category structure. usual notion of colimits cannot be correct for topological structure. 

$\mathrm{colim } : Top \to Top$ is not homotopical functor. 

We want to give a modern correct formulation for this. A cofibrant replacement in the correct sense. 

\subsection{Preliminaries}
\begin{definition}
    A left deformation of $\mathcal{C}$ is an endofunctor 
    \[
         Q: \mathcal{C}\to \mathcal{C}
    \] together with a natural weak equivalence  
    \[
         \eta: Q \xrightarrow[ ]{\simeq }\mathrm{ id }_\mathcal{C}
    \]
    $\mathcal{C}_{Q}$ full subcat of $\mathcal{X}$ that contains image of $Q$. 
\end{definition}
in model cat $\mathcal{C}_ Q$ is the category of cofibrant objects. 

\begin{definition}
    A left deformation of a functor $F: \mathcal{C} \to \mathcal{D}$. is a left derived functor $$q: Q\to \mathrm{id }_ \mathcal{C}$$ of $\mathcal{C}$ and a choice of $\mathcal{C}_Q$ such that $F$ is homotopical on $C_Q$. 
\end{definition}
\begin{example}
    If $M$ is a model cat. $F$ is left Quillen, then the cofibrant replacement is a left def of $F$ by Ken Brown lemma 
\end{example}
\begin{example}
    $M=CGWH$. 
    Fact: CW complexes are cofibrant. 
\end{example}
\begin{theorem}
     If $q: Q\implies id $ is a left def of $F: C\to D$, tehn 
     \[
         LF= FQ 
     \]is a left derived functor of F. 
\end{theorem}
\begin{proof}
     2.2.8 Riehl 
\end{proof}

\subsection{Homotopy colimts}
Let $J$ be small cat 
\begin{definition}
    the homotopy colimit functor, if it exists, is a left derived functor 
    \[
         \operatorname{hocolim}:=\mathbb{L}\operatorname{colim} : \mathcal{C}^J \to \mathcal{C}
    \] of $\operatorname{colim} :\mathcal{C}^J\to \mathcal{C}$. 

    To producte it find a left deformation functor $Q$ of $\mathcal{C}$ and consider $FQ$. 
\end{definition}

{\bf David's recap}\marginnote{L4 25/1}

We can compute the left derived functor of $F: C\to D$ between homotpical categories using a left def $Q: C\to C, q: Q\implies 1$ such that $F\vert_{\mathrm{im}  Q}$ is a homotpical functor by $\mathbb{L}F=FQ$. 

In $\mathop{\mathrm{Top }}, Q$ is CW approximation 
in $CH(R)$ $Q$ is projective resolution 

$\mathop{\mathrm{ colim }}: C^I \to C$ wher $C^I$ homotopical cat with pointwise weak equivalence . Then we copmute $\mathop{\mathrm{hocolim }}=\mathbb{L}\mathop{\mathrm{ colim }}$ using a bar complex. 

Homotopy pushouts: $\mathop{\mathrm{ hocolim}} Z\leftarrow X\to Y$ replace one of the morphisms with a cofibration. so the above is $\mathop{\mathrm{colim }}(QZ\leftarrow QX \to QY)$

you replace one of the objects with $Q(-)$ and then use the cofibration factorization on the other. But in top you only need one of $f$ and $g$ to be made into cofibration.   

\marginnote{L5 30/1}
\section{Stable phenomenon of Spaces }
Spaces of interest -$\mathrm{Top} $,$\mathrm{Top_{\ast}}$ 
These spaces arenot closed under products,even for CW complexes. $\mathrm{ CGWH } $ is a cateogry that satisfies these properties and we take these to mean spaces. 

\begin{definition}
  $(X,\cdot),(Y,\cdot)\in \mathrm{Top_{\ast }} $ Define $Y^X$ to be maps and $F(X,Y)$to be pointed maps equipped with compactopen topology. 

  Smash Product defined. 
\end{definition}

  $(\mathrm{Top_{\ast} } ,\wedge, S^0)$ make a closedsymmetric monoidal category. Internal $\hom=F(X,Y)$. 

  $\hom_{\mathrm{Top } }(X\wedge Y,Z)\cong \hom(X,F(Y,Z))$. 
 $F(X\wedge Y ,Z)\cong F(X,(Y,Z))$. 


$\pi_0(F(X,Y),\ast)\cong [S^0,F(X,Y)]\cong [X,Y]$

If we choose $Y=S^1$, then we get $F(\Sigma X,Z)\cong F(X,\Omega Z)$. If you apply yne $\pi_0$ functor we get homootpy classes of maps bijection.  in general $-\wedge Z \dashv F(-,Z)$

$\Sigma X$ is the pushout of $\ast \leftarrow X\to \ast$. 

$\Omega X$ is the pullback of $\ast \to X\leftarrow \ast$. 

$\Omega X$ is an $H$ group. $\Sigma X$ gives a co $H$ group structure. 

If $Y$ is a $H$ group then $[-,Y]$ is a group. If $X$ is a co gropu then $[X,Y]$ is a group. So the loopspace suspension adjunction is an isomorphism. 

If $[X,y]$ for $X$ co H group and $Y$ $H$ group, then $[X,Y]$ is an abelian group. 

Freudenthal Suspension Theorem 

$\Sigma :\pi_n(X)\to \pi_{n+1 }(\Sigma X)$ is an isomorphism for $0\leq i\leq 2n$ and epimorphism for $i=2n+1$. 

$\mathop{\mathrm{colim }}(\pi_{i+n+1 (\Sigma^n X)})=\pi_i^s(X)$ is the $i^{\text{th}  }$ stablehomotpy group. If we take $X=S^^0$, it is called 
$i$th stable stem. 

$\pi^*(S)$ is a graded ring, with graded commutativestructure\dots

$X\to \Omega \Sigma X$, is $(2n+1)$ connected if $X$ is $n$-connected. 

Define $QX=\mathop{\mathrm{colim }}(X\to \Omega \Sigma X\to (\Omega \Sigma)^2X\to \cdots )$ $\pi_*(QX)=\pi_*^s(X)$. 

Excision does nothold for $X=S^2,A=X\N, B=X\S$ and $A\cap B=S^1$. then $\pi_3(A,A\cap B)\cong \pi_2(S^1)=0$ and $\pi_3(S^2, B)\cong \pi_3(S^2)\neq 0$. 

Excision holds for higher homtopy groups upto connectivity. 

\begin{theorem}[Blaker's Massey Excision ]
If $X=A\cup B$, $A\cap B\neq \varphi$ and $A, C$ is $m$ connected and $(B,C)$ is $n$ connected
\[
  \pi_i(A,C)\xrightarrow[ ]{i_* }\pi_i(X,B)
\] isomorphism for $i<m+n$ and surjective for $i=m+n$. 
\end{theorem}

\begin{theorem}
     If 
     \[
      \begin{tikzcd}
Q \arrow[dd, "m-connected"'] \arrow[rr, "n-connected"] \arrow[rd, "h\, m+n-1 connected" description, dashed] &                          & Y \arrow[dd] \\
                                                                                                          & F \arrow[ld] \arrow[ru] &              \\
X \arrow[rr]                                                                                              &                          & P           
\end{tikzcd}
     \] here $F$ is the pullback of the inner square. his gives a version of van kampen. 

\end{theorem}
\begin{corollary}
  $X=\ast$. $Q\xrightarrow[ ]{g }Y \xrightarrow[ ]{c }P$ $Q$ is $m$ connected and $g$, $n$ connected. Then $\mathop{\mathrm{hofib }}=F$ $Q\to F$ is $m+n$ connected 
\end{corollary}

If $X,Y$ are $p$, $q$, connected then $X\vee Y \to X\times Y$ is $p+q+1$ connected.

\begin{theorem}{Hurewicz Theorem }
  fsdafasjkdfhljkh
\end{theorem}
\end{document}